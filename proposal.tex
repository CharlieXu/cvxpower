\documentclass[12pt]{article}
\usepackage{graphicx,psfrag,amsmath,caption,subcaption}
\usepackage{fullpage}
\input defs.tex

\title{Dynamic Energy Management}
\author{
Stephen Boyd\thanks{
Electrical Engineering Department, Stanford University. 
\texttt{boyd@stanford.edu}}
}
\date{\today}
\bibliographystyle{alpha}

\begin{document}
\maketitle

\section{Background}

There are many optimizations that must occur
to effectively manage the electric power grid,
including Security-Constrained Unit Commitment (SCUC),
Security-Constrained Optimal Power Flow (SCOPF),
Economic Dispatch, Loss Minimization, Reactive Power Compensation, et.~al.
As the size and complexity of these optimization problems grow,
the need to solve these problems in a timely manner
with a larger number of features and data sets
becomes increasing challenging.

Alternating Direction Method of Multipliers (ADMM)
is an algorithm that utilizes a decomposition coordination method
whereby a large problem is decomposed into small local sub-problems
that are coordinated in a manner that solves the larger global problem.
ADMM blends the benefits of dual decomposition methods
and augmented Lagrangian methods (ALM)
to solve large scale optimization problems using a scalable approach,
based in part on discovering optimal dual variables (marginal prices).
This research work is focused on operationalizing the ADMM algorithm
to support production electric power management.

\section{Research Outline}
The proposed research is split across two years,
with the first year constraining the scope to a simplified
optimal power flow problem (convex only), solved via a centralzied method,
and extending the problem to the distributed setting in year two.
The following section describes more specific activities 
associated across each year.

\subsection{Year One}
In the first year
we will concentrate on developing a foundational open source 
Python package for dynamic energy management,
following the abstract ideas laid out in 
\cite{kraning2014dynamic},
and focussing entirely on the convex case.
We hope that the package will support multiple types of 
dynamic energy optimization problems, including day-ahead 15 minute 
planning, but also including hydro resource management, minute-by-minute
planning, and even optimal management of a hybrid vehicle.
This package will be based on CVXPY 
\cite{cvxpy},
a open source convex optimization package in Python.
\begin{enumerate}
\item \emph{Defining the classes and methods.}  The first task is to 
identify the mathematical structure of the dynamic energy problems 
and develop Python code to support the abstractions.
\item \emph{Implementing classes.}
We will carefully define the critical classes,
including generators, storage devices, transmission lines,
basic loads, smart loads including deferrable loads, curtailable loads, 
HVAC loads, and connections to other sources with time varying prices 
and availability.
We will include several classes of green energy sources, 
including wind and solar,
possibly coupled with local storage and/or generators.
\item \emph{Adding additional methods.}
We will define the critical methods used to connect device objects,
optimize energy consumption and generation over a time interval,
retrieve time varying optimal prices at a bus or other 
energy exchange point, and so on.
\item \emph{Benchmarking.}
We will work with MISO to benchmark and analyze the scalability of the 
centralized algorithm.
%\item \emph{Algorithm Correctness:}
%Time permitting;
%we may leverage Alloy or TLA+ to formally analyze the algorithm for correctness
%and better understand boundary conditions.
\item \emph{User interface.} We will work with MISO to develop simple UIs,
code to translate standard descriptions into our package code, possibly
via XML.
\end{enumerate}

\paragraph{Year One Research Scope.}
\begin{enumerate}
\item \emph{Convex problems only.}
Year one work will focus entirely on convex problems,
since in this case the optimization is clear,
and we can rely on CVXPY as the base optimization layer.
In particular, year one will focus on (real) power flow,
and not the more traditional voltage and current phasors,
or real and reactive power flows.
This model can be made to include many realistic power systems,
and will include, for example, variable generator cost,
generator minimum and maximum powers,
various types of smart loads, DC and AC transmission line losses,
storage energy and power limits,
generator ramp rate limits.

%Although the focus of this year one research is centered on 
%Optimal Power Flow,
MISO will be applying and evaluating ADMM / CVXPY
against a simplified (linearized) Security Constrained
Unit Commitment (SCUC) problem.
Although we do not expect any direct support for
this parallel SCUC problem research,
we anticipate some occasional advisory input and guidance
to help inform further evaluation of ADMM within a power 
optimization context.

\item \emph{Centralized solution method.}
Year one work will focus on a centralized CVXPY solver 
(which could be multi-core).
\end{enumerate}

\subsection{Year Two}

Year two will extend the foundational work created in year one
to potentially include exploration of the 
following areas and to help further operationalize the dynamic energy
management package.
\begin{itemize}
\item \emph{Distributed solution via ADMM.}  We will develop methods to 
automatically split up a large problem and solve it using ADMM.
\item \emph{Convex relaxation.} 
We will explore methods for handling
non-convex devices and objectives, 
including for example on/off loads and generators,
AC power flow constraints on loop networks, 
and other non-convexities.
\item \emph{Closed loop control.} 
We will explore using receding horizon control (RHC) 
\cite{mattingley2011receding,mattingley2009automatic}
or model predictive control (MPC) 
\cite{wang2010fast}
to adapt schedules in response to external disturbances adapting the
solution across an evolving time horizon by incorporating new 
information as it becomes available.
\item \emph{Security constraints.} 
We will explore the incorporation of security constraints
directly in the optimization, 
by adding a set of contingencies for devices and carrying out
planning for operation under nominal and contingent operations.
\item \emph{Hierarchical extensions.}
We will explore the extension of the model to support the 
natural hierarchy of the power grid 
by possibly scheduling messages on different time scales 
between systems of similar hierarchy.
\end{itemize}

\subsection{Future Extensions}

The following section outlines additional areas of investigation
for operationalizing the ADMM algorithm.
These items are considered beyond the scope of this initial two-year effort.

\begin{enumerate}
\item \emph{Incentive compatible protocols.}
Modifying (if needed) the core ADMM algorithm to be incentive compatible,
so the system is robust to
non-cooperative or antagonistic devices.  This can include bad behavior
detection and counter measures to ensure proper system behavior.
\end{enumerate}

\section{Approach}

\begin{enumerate}
\item 
We will develop a framework for running such a system in model predictive control (MPC)
mode, using predictions of future values (loads, prices, etc.) over a horizon.

\item
We will publish the code under GitHub,
with full documentation. We will also prepare a paper to be published on the project.
All code will be released under a GNU 3.0 license.

\item 
We will work with MISO Energy technical contacts to keep them in the loop,
ensure that the code we write can be run in their environment,
and get general feedback on the organization with short updates
occurring on a monthly basis.

\item \emph{Travel.} 
Stephen Boyd and the assigned graduate student
will travel to MISO once each year and the assigned MISO technical 
resources travel to Stanford a minimum of twice each year.

%\item \emph{Exit criteria,} At the end of the first year,
%either Stanford or MISO Energy may decide to
%terminate this research agreement for any reason.
%
%\item \emph{Payment,} MISO will pay for each year in whole at the start of that year.
\end{enumerate}

\bibliography{proposal}

\end{document}
