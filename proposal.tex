\documentclass[12pt]{article}
\usepackage{graphicx,psfrag,amsmath,caption,subcaption}
\usepackage{fullpage}
\input defs.tex

\title{Dynamic Energy Management}
\author{
Stephen Boyd\thanks{
Electrical Engineering Department, Stanford University. 
\texttt{boyd@stanford.edu}}
}
\date{\today}
\bibliographystyle{alpha}

\begin{document}
\maketitle

\section{Background}

There are many optimizations that must occur
to effectively manage the electric power grid,
including Security-Constrained Unit Commitment (SCUC),
Security-Constrained Optimal Power Flow (SCOPF),
Economic Dispatch, Loss Minimization, Reactive Power Compensation, et.~al.
As the size and complexity of these optimization problems grow,
the need to solve these problems in a timely manner
with a larger number of features and data sets
becomes increasing challenging.
Alternating Direction Method of Multipliers (ADMM)
is an algorithm that utilizes a decomposition coordination method
whereby a large problem is decomposed into small local sub-problems
that are coordinated in a manner that solves the larger global solution.
ADMM blends the benefits of Dual Decomposition methods
and Augmented Lagrangian Methods (ALM)
to solve large scale optimization problems using a scalable approach.
This research work is focused on operationalizing the ADMM algorithm
to support production electric power management.

\section{Research Outline}
The proposed research is split across two years,
with the first year constraining the scope to a simplified
optimal power flow problem (convex only)
and extending the problem in complexity and scope in year two.
The following section describes more specific activities associated across each year:

\subsection{Year One}
In the first year,
we will concentrate on developing a foundational open source Python package
for dynamic energy management,
following the abstract ideas laid out in \cite{kraning2014dynamic}.
This research will focus on extending CVXPY in the following manner:
\begin{enumerate}
\item \emph{Adding Additional Classes:}
We will carefully define the critical classes,
including generators, storage devices, transmission lines,
basic loads, smart loads including deferrable loads, curtailable loads, HVAC loads,
and connections to other sources with time varying prices and availability.
We will include several classes of green energy sources, including wind and solar,
possibly coupled with local storage and/or generators.

\item \emph{Adding Additional Methods:}
We will define the critical methods used to connect device objects,
optimize energy consumption and generation over a time interval,
retrieve time varying prices at a bus, and so on.

\item \emph{Benchmarking:}
MISO will work to benchmark and analyze the scalability of the algorithm
in our labs---we will leverage the resulting information to help inform further analysis.

\item \emph{Algorithm Correctness:}
Time permitting;
we may leverage Alloy or TLA+ to formally analyze the algorithm for correctness
and better understand boundary conditions.
\end{enumerate}

\paragraph{Year One Research Scope.}
\begin{enumerate}
\item \emph{Convex Problems Only:}
Year one work will focus entirely on convex problems,
since in this case the optimization is clear,
and we can rely on CVXPY as the base optimization layer.

\item \emph{Optimal Power Flow:}
Year one will focus on power flow,
and not the more traditional voltage and current phasors,
or real and reactive power flows.
This model can be made to include many realistic power systems,
and will include, for example, variable generator cost,
generator minimum and maximum powers,
various types of smart loads, DC and AC transmission line losses,
storage energy and power limits,
generator ramp rate limits.
Although the focus of this year one research is centered on Optimal Power Flow,
MISO will also be applying and evaluating ADMM / CVXPY
against a simplified (linearized) Security Constrained
Unit Commitment (SCUC) problem set as well.
Although we do not expect any direct support for
this parallel SCUC problem research,
we anticipate some occasional advisory input and guidance
to help inform further evaluation of ADMM within a power optimization context.

\end{enumerate}

\subsection{Year Two}

Year two will extend the foundational work created in year one
to potentially include exploration of the 
following areas and to help further operationalize ADMM:
\begin{itemize}
\item \emph{Convex Relaxation:} 
finding solutions when including non-convex devices and objectives, 
including for example on/off loads and generators,
AC power flow constraints on loop networks, 
and other non-convexities.

\item \emph{Closed Loop Control:} 
adjusting schedules in response to external disturbances adapting the
solution across an evolving time horizon by incorporating new information as it becomes
available.

\item \emph{Security Constraints:} 
adding a set of contingencies for devices connected to the electrical
power network for operation under nominal and contingent operations
 
\item \emph{Hierarchical Extensions:}
extend the model to support the natural hierarchy of the power grid 
by possibly scheduling messages on different time scales between systems of similar hierarchy.
\end{itemize}

\subsection{Future Extensions}

The following section outlines additional areas of investigation
for operationalizing the ADMM algorithm.
These items are considered beyond the scope of this initial two-year effort:

\begin{enumerate}
\item \emph{Incentive Compatible Protocols}
examination of secure and reliable behavior of nodes to
ensure cooperative behavior,
including non-cooperative or antagonistic devices or bad behavior
detection and counter measures to ensure proper system behavior.

\item \emph{Local Stopping Criteria}
current stopping criteria require global device coordination---this
would explore a decentralized approach leveraging gossip or epidemic algorithms, letting
individual devices decide stopping criteria and values of $\rho$.
\end{enumerate}

\section{Approach}

\begin{enumerate}
\item 
We will develop a framework for running such a system in model predictive control (MPC)
mode, using predictions of future values (loads, prices, etc.) over a horizon.

\item
We will publish the code under GitHub,
with full documentation. We will also prepare a paper to be published on the project.

\item 
We will work with MISO Energy technical contacts to keep them in the loop,
ensure that the code we write can be run in their environment,
and get general feedback on the organization with updates
occurring---we will make every effort to touch base on a monthly basis.

\item \emph{Travel:} Stephen Boyd and the assigned graduate student
will travel to MISO once each year and the assigned MISO technical resources travel
to Stanford a minimum of twice each year.

\item \emph{Exit Criteria:} at the end of the first year,
either Stanford or MISO Energy may decide to
terminate this research agreement for any reason.

\item \emph{Payment:} MISO will pay for each year in whole at the start of that year.
\end{enumerate}

\bibliography{proposal}


\end{document}


